%
% Tesis de Ingenieria en Electronica.
%

\documentclass[twoadviser,master]{labicetesis}

%======================== Directorios de trabajo =============================

\newcommand{\DirFigF}{./Figures/FrontPage}

%=============================================================================

\usepackage{pdfpages}
\usepackage{listings}

%========================    Datos personales    =============================

\title{Desarrollo de un asistente virtual por reentrenamiento de LLMs con recuperación-generación aumentada desde documentos normativos}		% Titulo del documento.
\author{Ing. Roberto García Guzmán}				% Autor del documento.
\date{Diciembre, 2025}									% Fecha de publicacion.
\crest{\DirFigF/escudo-bn}				% Escudo de la Universidad
\grade{MAESTRO EN INGENIERÍA ELÉCTRICA}			% Grado o Titulo
\adviser{Dra. Dora Luz Almanza Ojeca}{Dr. Yair Alejandro Andrade Ambriz}	%Asesor(es)
\bars{\DirFigF/bars}									%Barras

\hyphenation{}
\clubpenalty=10000
\widowpenalty=10000

%=========================    Documento          =============================
\begin{document}                        % Empieza el documento
\pagenumbering{roman}
\maketitle			% Portada
\formatLabice			% Diseño de la hoja y Formato de tesis
%\input{./Chapters/agradecimientos2}   % Agradecimientos Institucionales.
%\input{./Chapters/agradecimientos1}   % Agradecimientos.
%\input{./Chapters/Abstract}			  % Abstract

\dominitoc
\tableofcontents                      % Índice general
%\listoffigures                        % Índice de figuras
%\listoftables                         % Índice de tablas
\newpage


%==========================      Capítulos     ================================

\pagenumbering{arabic}
\parskip = 0.5cm			%espacio entre parrafos
\chapter{Introducción}
\minitoc
\label{Cap:Int}

\cite{sample-article}

\appendix
%\input{./Chapters/ApendiceA}

%========================== Bibliography ======================================
\bibliographystyle{IEEEtran}
\bibliography{./Bib/Bibliography}
\addcontentsline{toc}{chapter}{Bibliografía}

\end{document}