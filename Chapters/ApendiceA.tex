\chapter{Limitaciones convertidores de PDF a texto}

\section{Fromato PDF}

El formato PDF (Portable Document Format, en inglés) es un formato cuyo objetvio
es ofrecer una forma sencilla y segura de presentar e intercambiar documentos con
independencia del software, el hardware o el sistema operativo que utilice quien
los consulte [cita Adobe]. Fue creado por Adobe en 1993 y posteriormente estandarizado por la
ISO (Organización Internacional de Normalización) bajo la ISO 32000-1:2008
(PDF 1.7) y más recientemente la ISO 32000-2:2020 (PDF 2.0) [cita ISO].

Gracias a su estándarización, el formato PDF es el medio \textit{de facto} para
compartir documentos entre dispositivos y plataformas, ya que existen múltiples
herramientas de lectura de archivos PDF y una gran cantidad de editores de texto
o imágenes cuentan con capacidades de exportación a PDF.

Sin embargo, el formato PDF fue pensado como una herramienta para presentar e intercambiar
documentos, entendiendo la presentación como el hecho de que un humano vea e interprete
el documento sin dificultad, ya sea de forma digital o impresa, es decir, no está
diseñado para que una máquina interprete su contenido de forma fácil, sino para que
un humano lo haga.

Al emplear este formato, pensado para la visualización por parte de un humano, el formato
presenta dificultades cuando se requiere extraer la información textual que contiene,
lo cual es necesario para cualquier tarea donde se empleen modelos grandes de lenguaje.

Existen diferentes herramientas para la extracción de texto desde archivos PDF, las cuales
presentan errores errores o limitaciones que son propias del formato PDF o relacionadas
con el algoritmo de extracción empleado por la herramienta.

Se hizo un análisis de diferentes convertidores de PDF a texto, especialmente de
aquellos cuyas derivaciones se usan en frameworks para el uso de modelos de lenguaje,
siendo los principales los siguientes:

\begin{itemize}
      \item \textbf{pypdf}: Libería de código abierto escrita puramente en Python para manipular
            archivos PDF. \url{https://pypdf.readthedocs.io/en/stable/}
      \item \textbf{pdftotext}: Herramienta de línea de comandos de Linux. Incluida en
            la mayoría de las distribuciones con el paquete poppler-utils.
            \url{https://linux.die.net/man/1/pdftotext}
      \item \textbf{pdfplumber}: Libería de código abierto de Python para sondear información
            detallada un documento.
            \url{https://github.com/jsvine/pdfplumber?tab=readme-ov-file}
      \item \textbf{pymupdf4llm}: Libería de código abierto de Python (basada en pymupdf) para
            extraer información de archivos PDF y convertirlo a formato md u otros.
            \url{https://pymupdf.readthedocs.io/en/latest/tutorial.html}
\end{itemize}

\begin{table}
      \begin{tabularx}{0.85\textwidth}{|X|c|c|c|c|}
            \hline
            \textbf{Característica}                   & \textbf{pypdf} & \textbf{pdftotext} & \textbf{pdfplumber} & \textbf{pymupdf4llm} \\
            \hline
            Detección de títulos y subtítulos         & \xmark**       & \xmark             & \cmark***           & \xmark               \\
            Detección de subtítulos en columnas       & \xmark**       & \xmark             & \cmark***           & \xmark               \\
            Texto ordenado siempre                    & \xmark         & \xmark             & \cmark***           & \xmark               \\
            Detección de encabezados y pies de página & \cmark         & \xmark             & \cmark              & \xmark               \\
            Interpretación de índices                 & \cmark*        & \xmark             & \cmark              & \cmark               \\
            Texto a dos columnas                      & \cmark         & \cmark             & \cmark              & \cmark               \\
            Texto con tabulaciones intermedias        & \xmark         & \xmark             & \cmark***           & \xmark               \\
            Detección de tablas                       & \xmark         & \xmark             & \cmark              & \cmark               \\
            Detección de letra capital                & \xmark         & \xmark             & \cmark              & \xmark               \\
            Detección de texto duplicado              & \xmark         & \xmark             & \cmark              & \cmark               \\
            \hline
      \end{tabularx}
      \caption{* La librería ordena correctamente el texto pero no es consciente de que se trata de un índice.
            ** La librería proporciona algunas formas de optener las posiciones, el tipo y tamaño de letra pero no es fácil interpretarlas.
            *** Al usar directamente la función de extracción de texto de la librería se encuentran estos errores, pero tiene la funcionalidad de regresar el boundingbox de cada palabra.}
\end{table}

PdfPlumber es una librería muy completa que incluso proporciona los boundingboxes de cada palabra, sin embargo,
sus funciones para procesar texto a más de una columna son deficientes. También proporciona una función
para extraer las tablas del documento.

