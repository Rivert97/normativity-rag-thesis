\chapter{Convertidores de PDF a texto}

Se realizó un análisis de diferentes convertidores de PDF a texto,
especialmente aquellos cuyas derivaciones se usan en frameworks para el
uso de modelos de lenguaje, siendo los principales los siguientes:

\begin{itemize}
      \item \textbf{pypdf}: Libería de código abierto escrita puramente en Python para manipular
            archivos PDF. \url{https://pypdf.readthedocs.io/en/stable/}
      \item \textbf{pdftotext}: Herramienta de línea de comandos de Linux. Incluida en
            la mayoría de las distribuciones con el paquete poppler-utils.
            \url{https://linux.die.net/man/1/pdftotext}
      \item \textbf{pdfplumber}: Libería de código abierto de Python para sondear información
            detallada un documento.
            \url{https://github.com/jsvine/pdfplumber?tab=readme-ov-file}
      \item \textbf{pymupdf4llm}: Libería de código abierto de Python (basada en pymupdf) para
            extraer información de archivos PDF y convertirlo a formato md u otros.
            \url{https://pymupdf.readthedocs.io/en/latest/tutorial.html}
\end{itemize}

En la tabla se muestran con (*) aquellas librerías que ordenan correctamente
el texto pero no funciona con los índices. Se usan (**) en las librerías que
proporcionan algunas formas de obtener las posiciones del texto, tipo y
tamaño de letra pero es dificil interpretarlas. Los (***) muestran
librearías que, al usar directamente la función de extracción de texto, se
encuentra el error especificado, pero que proveen la funcionalidad de
regresar el recuadro que contiene cada palabra, por lo que, aplicando las
técnicas descritas en el documento se podrían solucionar.

\begin{table}
      \begin{tabularx}{0.85\textwidth}{|X|c|c|c|c|}
            \hline
            \textbf{Característica}                   & \textbf{pypdf} & \textbf{pdftotext} & \textbf{pdfplumber} & \textbf{pymupdf4llm} \\
            \hline
            Detección de títulos y subtítulos         & \xmark**       & \xmark             & \cmark***           & \xmark               \\
            \hline
            Detección de subtítulos en columnas       & \xmark**       & \xmark             & \cmark***           & \xmark               \\
            \hline
            Texto ordenado siempre                    & \xmark         & \xmark             & \cmark***           & \xmark               \\
            \hline
            Detección de encabezados y pies de página & \cmark         & \xmark             & \cmark              & \xmark               \\
            \hline
            Interpretación de índices                 & \cmark*        & \xmark             & \cmark              & \cmark               \\
            \hline
            Texto a dos columnas                      & \cmark         & \cmark             & \cmark              & \cmark               \\
            \hline
            Texto con tabulaciones intermedias        & \xmark         & \xmark             & \cmark***           & \xmark               \\
            \hline
            Detección de tablas                       & \xmark         & \xmark             & \cmark              & \cmark               \\
            \hline
            Detección de letra capital                & \xmark         & \xmark             & \cmark              & \xmark               \\
            \hline
            Detección de texto duplicado              & \xmark         & \xmark             & \cmark              & \cmark               \\
            \hline
      \end{tabularx}
      \caption{Tabla comparativa de errores y funcionalidades de extractores de texto de
            archivos PDF.}
\end{table}

