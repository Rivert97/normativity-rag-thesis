\chapter{Limitaciones convertidores de PDF a texto}

\section{Fromato PDF}

Se realizó un análisis de diferentes convertidores de PDF a texto,
especialmente aquellos cuyas derivaciones se usan en frameworks para el
uso de modelos de lenguaje, siendo los principales los siguientes:

\begin{itemize}
      \item \textbf{pypdf}: Libería de código abierto escrita puramente en Python para manipular
            archivos PDF. \url{https://pypdf.readthedocs.io/en/stable/}
      \item \textbf{pdftotext}: Herramienta de línea de comandos de Linux. Incluida en
            la mayoría de las distribuciones con el paquete poppler-utils.
            \url{https://linux.die.net/man/1/pdftotext}
      \item \textbf{pdfplumber}: Libería de código abierto de Python para sondear información
            detallada un documento.
            \url{https://github.com/jsvine/pdfplumber?tab=readme-ov-file}
      \item \textbf{pymupdf4llm}: Libería de código abierto de Python (basada en pymupdf) para
            extraer información de archivos PDF y convertirlo a formato md u otros.
            \url{https://pymupdf.readthedocs.io/en/latest/tutorial.html}
\end{itemize}

\begin{table}
      \begin{tabularx}{0.85\textwidth}{|X|c|c|c|c|}
            \hline
            \textbf{Característica}                   & \textbf{pypdf} & \textbf{pdftotext} & \textbf{pdfplumber} & \textbf{pymupdf4llm} \\
            \hline
            Detección de títulos y subtítulos         & \xmark**       & \xmark             & \cmark***           & \xmark               \\
            Detección de subtítulos en columnas       & \xmark**       & \xmark             & \cmark***           & \xmark               \\
            Texto ordenado siempre                    & \xmark         & \xmark             & \cmark***           & \xmark               \\
            Detección de encabezados y pies de página & \cmark         & \xmark             & \cmark              & \xmark               \\
            Interpretación de índices                 & \cmark*        & \xmark             & \cmark              & \cmark               \\
            Texto a dos columnas                      & \cmark         & \cmark             & \cmark              & \cmark               \\
            Texto con tabulaciones intermedias        & \xmark         & \xmark             & \cmark***           & \xmark               \\
            Detección de tablas                       & \xmark         & \xmark             & \cmark              & \cmark               \\
            Detección de letra capital                & \xmark         & \xmark             & \cmark              & \xmark               \\
            Detección de texto duplicado              & \xmark         & \xmark             & \cmark              & \cmark               \\
            \hline
      \end{tabularx}
      \caption{* La librería ordena correctamente el texto pero no es consciente de que se trata de un índice.
            ** La librería proporciona algunas formas de optener las posiciones, el tipo y tamaño de letra pero no es fácil interpretarlas.
            *** Al usar directamente la función de extracción de texto de la librería se encuentran estos errores, pero tiene la funcionalidad de regresar el boundingbox de cada palabra.}
\end{table}

